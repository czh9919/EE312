\documentclass[12pt,a4paper]{article}
\usepackage[]{listings}
%packages used
\usepackage[english]{babel}
\usepackage[margin=1in,  left=1.25in]{geometry} %margins
\usepackage{pslatex}%Times new roman font
\usepackage{graphicx}%for images
\usepackage{float}
%\usepackage{fancyhdr}
%\pagestyle{fancy}
%\fancyfoot{}

%beginning of document
\begin{document}
%declaration page
%\thispagestyle{empty}
\pagenumbering{roman} 
\begin{titlepage}
  \begin{center}
    \vspace*{1cm}

    \textbf{\Huge ALU project report}

    \vspace{0.5cm}

         
    \vspace{1.5cm}

    \textbf{\large Chang Zihao \\20206018\\\large Cui Yuxuan\\20206019}

    \vfill
         

         
    \vspace{0.8cm}
  


         
\end{center}
\end{titlepage}


\newpage
%table of contents
\tableofcontents
\thispagestyle{empty}

\newpage
\pagenumbering{arabic}
\setcounter{page}{1}

\section{Introduction}

A Finite State Machine (FSM) is a machine that can be in exactly one of a finite number of states at any given time. 
The FSM can transit from one state to another state in response to external inputs; 
the transition between states is predefined in the FSM. 
An FSM is defined with a list of state, conditions for each transition, and initial state. 
FSMs are of two types of FSM, which are the Moore machine and Mealy machine. 
A Moore machine is an FSM whose outputs are determined only by its current state. 
This is in contrast to a Mealy machine whose outputs are determined by both its current state and inputs. 
We will use these two structures to design a vending machine based on these two structures.


\newpage

\section{Design}
\subsection{Module design}

For design an vending machine, we should know how the vending machine works.
This vending machine is a sample machine that based on the FSM.
The process of the machine is that input coins and then choose items like drinking, or water, then it will give changes and output items.
And in the code we devided in three parts. 
The first parts is input coins, this part will gets the coin number changes and the total coin value changes, then select items changes.
The second parts when a state change is detected, this part will start to operate, ready for output
The third part is at the posedge, update the data, and ready to return the coins. 


\begin{figure}[H]
  \centering
  \includegraphics[height=1in]{circuit.jpg}
  \end{figure}


\newpage

\section{Implementation}

Like we said in the design part, we devided the code three part,
First set 0 at the beginning, all inputs must set to 0, because if we don't set 0, all inputs will in a high resit
Then input the coin. There are three kinds of coins 100,500,1000.
i_input_coin are three bits binary number. 
Then input i_input_coin, select the corresponding value coin, then coin number plus 1, total value increase corresponding coin value.
Similar with the input coin select items is also use three bits binary number to control the items.
Then the number of the item plus 1.

And for the second part is to output the items. 
After we select items, then the number of the items are changed, at the posedge it will output the items, one item at a time.

The third part is the most important.
At the posedge if we choose reset, all variable should be set 0.
And else case we first update the value, let num_coins =num_coins + num_coins_nxt.
Then determine which items can we buy with the value of the coin.
At last we must calculate the remain value of the coin, then return the coin.
The method is, first determine how much the coin remained, and compare with the coin value.
If it more than ot equal to 1000 then first return a 1000 coin, then the remained value is more than 500 then return a 500 coin and so on.
Then a cycle is done.  


\newpage

\section{Evaluation}



\newpage
\section{Discussion}



\section{Conclusion}



\end{document}
