\documentclass[12pt,a4paper]{article}
\usepackage[]{listings}
%packages used
\usepackage[english]{babel}
\usepackage[margin=1in,  left=1.25in]{geometry} %margins
\usepackage{pslatex}%Times new roman font
\usepackage{graphicx}%for images
\usepackage{float}
%\usepackage{fancyhdr}
%\pagestyle{fancy}
%\fancyfoot{}

%beginning of document
\begin{document}
%declaration page
%\thispagestyle{empty}
\pagenumbering{roman} 
\begin{titlepage}
  \begin{center}
    \vspace*{1cm}

    \textbf{\Huge Single cycle CPU  report}

    \vspace{0.5cm}

         
    \vspace{1.5cm}

    \textbf{\large Chang Zihao \\20206018\\\large Cui Yuxuan\\20206019}

    \vfill
         

         
    \vspace{0.8cm}
  


         
\end{center}
\end{titlepage}


\newpage
%table of contents
\tableofcontents
\thispagestyle{empty}

\newpage
\pagenumbering{arabic}
\setcounter{page}{1}

\section{Introduction}
Lab 3 is intended to give you hands-on experience in designing the simplest form of a modern microprocessor, whose operations are conducted within a single clock cycle. 
Based on the concepts and skills you have acquired through Lab 1 and Lab 2, you are now ready to implement a single-cycle CPU. 
Through this lab assignment, you will have gained an in-depth understanding on the fundamentals of designing a CPU microarchitecture.  
  
2. Backgrounds 
Control Path & Data Path 
A CPU is composed of two types of units, which are the control units and data units. 
A control path consists of multiple control units, which “controls” the data path of the processor (e.g., ALUs, memory) how to respond to the instruction that is fetched to the processor. 
Likewise, a data path is a collection of functional units such as arithmetic logic units (ALUs) or multipliers, that perform data processing operations, registers, and buses. 
You can simply think of the data path as it literally means; the path where the data flow inside of the processor. 
 
Single-cycle CPU 
A single-cycle CPU processes an instruction per every single clock cycle. 
At every single clock cycle, the singlecycle CPU must perform Instruction Fetch (IF), Instruction Decode (ID), Execute (EX), Memory operation (MEM), and Write Back (WB) stages of a single instruction, so the architectural states are updated as the instruction “instructs” the CPU. 
The IF stage is to fetch the instruction that the Program Counter (PC) points to in the instruction memory. 
The ID stage is to decode the fetched instruction, figure out which operations are required to carry out, which is defined by the Instruction Set Architecture (ISA). 
The EX stage is where primitive operations (e.g., add, multiply, bit-wise operations) as well as memory address calculations are performed. 
The MEM stage is to perform memory operations (e.g., load, store) on specified address by the instruction. 
The WB stage is to update the architectural states of the processor; the values of registers or memory are updated as a result of executing the instruction. 
The exact control and data path are shown in Fig. 1, which is the MIPS version of a single cycle CPU design. 
Remember that the goal of Lab 3 is to design your own "RISC-V" version of such single cycle CPU microarchitecture. 





\subsection{coin-operating}

When someone puts in the coin, the vending machine records the number of changed coins, so that the number of coins updated can be recorded in this clock cycle.

\subsection{checking available items}

Once the coins changed in one cycle, the available items will be shows up.

\subsection{selecting items}

When someone chooses an item, the vending machine needs to first check whether the coins are sufficient, and then record it. 
If there is no problem, the item is output in the next clock cycle.

\subsection{refunding coins}

My idea is that after each coin is thrown, after selecting the item, the remaining amount of money is calculated and the change is checked, and the currency is ready to be refunded. 
Every clock cycle requires the currency to be refunded.
When you need to refund coins, refund one at one time.

\newpage

\section{Design}
\subsection{Module design}

To design a vending machine, we should know how the vending machine works.
This vending machine is a sample machine based on the FSM.
The process of the machine is that input coins and then choose items like drinking, or water, then it will give changes and output items.
And in the code, we divided into three modules. 

\subsection{input module}
The input module is input coins, this part will get the coin number changes and the coin value changes, and select items changes.

\subsection{output module}
The output module is when a state change is detected, this part will start to operate, ready for output.

\subsection{record module}

The third part is at the posedge, update the data, and ready to return the coins. 
Every record operation only can be done at this time.
Once finished, the vending machine will jump to output module if needs.

% \begin{figure}[H]
%   \centering
%   \includegraphics[height=1in]{circuit.jpg}
%   \end{figure}


\newpage

\section{Implementation}

Like we said in the design part, we divided the code into three-part.

\subsection{input module}

First set 0 at the beginning, all inputs must set to 0, because if we don't set 0, all inputs will in a high resit
Then input the coin. There are three kinds of coins 100,500,1000.
$i\_input\_coin $are three bits binary number. 
Then input $i\_input\_coin$, select the corresponding value coin, then coin number plus 1, the total value increase the corresponding coin value.
Similar to the input coin select items are also used three bits binary number to control the items. 
And determine which item can we buy.
Then the number of the item plus 1.
I used the case statement and if statement here.
Both of them are fine for the feature.

\subsection{output module}

And for the second part is to output the items. 
After we select items, then the number of the items is changed, after the posedge pass, it will output the items, one item at a time.

\subsection{record module}

The third part is the most important.
At the posedge if we choose to reset, all variables should be set 0.
And all inputs must set to 0, because if we don't set 0, all inputs will in a high resit
And else case we first update the value, let $num\_coins =num\_coins + num\_coins\_nxt$.
Then update which items can we buy with the value of the coin.
At last, we must calculate the remaining value of the coin, then return the coin.
The method is, first determine how much the coin remained, and compare it with the coin value.
If it more than or equal to 1000 then first return a 1000 coin, then the remained value is more than 500 then return a 500 coin and so on.
Then a cycle is done.  


\newpage

\section{Evaluation}

We pass all the tests except "ItemTest, CoinTest1, and CoinTest2".

\lstset{language=Verilog}
\begin{lstlisting}
# TEST                       InitialTest :
# PASSED
# TEST                 Insert100CoinTest :
# PASSED
# TEST                 Insert500CoinTest :
# PASSED
# TEST                Insert1000CoinTest :
# PASSED
# TEST                 Select1stItemTest :
# PASSED
# TEST                 Select2ndItemTest :
# PASSED
# TEST                 Select3rdItemTest :
# PASSED
# TEST                 Select4thItemTest :
# PASSED
# TEST                    WaitReturnTest :
# PASSED
# TEST                 TriggerReturnTest :
# PASSED
# Passed = 10, Failed = 0
# 1
\end{lstlisting}

\newpage
\section{Discussion}

We first encountered some problems in understanding the meaning of the question, 
but in the end by looking at the test code, 
we thoroughly understood the meaning of the question.
And at last, we pass all the test and finished the whole project.
I hope that future projects will give enough time to think and complete the code.

\section{Conclusion}

We implement the basic vending machine.
After that, I think we totally understood the FSM, how it works, and how to make it works.
To finish this project in limit time is a challenge.

\end{document}
